\documentclass[10pt]{article}
\usepackage[letterpaper]{geometry}
\geometry{verbose,tmargin=0.9in,bmargin=0.9in,lmargin=0.9in,rmargin=0.9in}
\usepackage{setspace}
\usepackage{titlesec}
\usepackage{graphicx}
\usepackage{float}
\usepackage{mathtools}
\usepackage{amsmath}
\usepackage[font=small,labelfont=bf,labelsep=period]{caption}
\usepackage[english]{babel}
\usepackage{indentfirst}
\usepackage{array}
\usepackage{makecell}
\usepackage[usenames,dvipsnames]{xcolor}
\usepackage{multirow}
\usepackage{tabularx}
\usepackage{arydshln}
\usepackage{caption}
\usepackage{subcaption}
\usepackage{xfrac}
\usepackage[numbers,sort&compress]{natbib}
\usepackage{enumitem}
\setlength{\bibsep}{0pt}

% Use "todonotes" - ability to add comments in the margins
\usepackage{todonotes}

% Set up hyperlinks
\usepackage[colorlinks=true, citecolor=Blue, linkcolor=Blue, urlcolor=Blue]{hyperref}
% example usage --> \todo[author=DH, linecolor=blue!40, backgroundcolor=green!20!white, bordercolor=black, size=\small]{This is a comment}

% Set up section\subsection title formats
\renewcommand{\thesection}{\Roman{section}} 
\renewcommand{\thesubsection}{\thesection.\Roman{subsection}}
\titleformat*{\section}{\normalsize\bfseries}
\titleformat*{\subsection}{\normalsize\bfseries}

\begin{document}


% = = = o = = = o = = = o = = = o = = = o = = = o = = = o = = =
%       TITLE & AUTHORS
% = = = o = = = o = = = o = = = o = = = o = = = o = = = o = = =

\begin{centering}
\textbf{Geant4 Simulation Package for the Portable Radiation Imaging Spectroscopy and Mapping (PRISM) System} \\
%\vspace{10pt}
%D. Hellfeld\textsuperscript{1,2} \\ 
%\vspace{5pt}
%\emph{\textsuperscript{1}Department of Nuclear Engineering, University of California, Berkeley, Berkeley, CA 94720} \\
%\emph{\textsuperscript{2}Applied Nuclear Physics, Lawrence Berkeley National Laboratory, Berkeley, CA 94720}  \\ 
\vspace{10pt}
Updated: \today \\
\end{centering}



% = = = o = = = o = = = o = = = o = = = o = = = o = = = o = = =
%       General Remarks
% = = = o = = = o = = = o = = = o = = = o = = = o = = = o = = =

\section{General Remarks}

\begin{itemize}
	\item{Detectors are CZT, world is vacuum}
	\item{Basic physics is included (Compton, Photoelectric, Pair Production, Rayleigh..)}
	\item{If photon hits detector and has photoabsorption reaction, we tally into ROOT histogram}
	\begin{itemize}
  		\item{ROOT histogram can then be read by python script in the analysis directory}
  		\item{another python script is used to perform the reconstruction}
		\begin{itemize}
			\item{from within the analysis directory run \$ ipython}
			\item{once in ipython, run \$ run readsystemresponse.py, \$ run reconstruction.py}
		\end{itemize}
	\end{itemize}
	\item{can be run by running \$ ./CodedAperture}
	\begin{itemize}
 		\item{macros exist that will run particles at every angle in -30,30 in theta and phi}
    		\item{see the \texttt{response\_loop} macros}
  		\item{update this to use HEALpix angles�}
	\end{itemize}
	\item{output will be in the form of a root file names totalresponse.root}
	\item{to build with cmake use the following:\\ \$ cmake \texttt{-DGeant4\_DIR=/path/to/geant4-install/lib/Geant4-10.0.2/} . }
	\begin{itemize}
		\item{then just use \$ make -j4}
	\end{itemize}
	\item{or if you want to use Xcode, use the "-G Xcode" flag}
\end{itemize}


% = = = o = = = o = = = o = = = o = = = o = = = o = = = o = = =
%       Details
% = = = o = = = o = = = o = = = o = = = o = = = o = = = o = = =

\section{Details}




\end{document}


